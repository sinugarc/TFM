\cleardoublepage

\chapter{Introduction}
\label{Ch1:Intro}
QuraTest 8/11/23: Sí que utiliza mutantes, de hecho llega a generar más de 20k programas mutantes, pero para ello usa QMutPy de Rui et al. Se enfoca en la generación efectiva de test. (No está en la lista que obtuve por esas fechas, lo añadiré)
QuCat 8/11/23: Es solo el resumen de una conferencia de septiembre de 2023, aunque ya tenía el original datado de 2021-2022, no añade nada respecto al anterior. Si que es cierto que en related work sólo menciona como papers sobre mutaciones el QMutPy de Rui y el MutTG, que es suyo, y también lo tenemos ya. 
NoiseAwareness 15/1/24: No hace nada con mutaciones pero si es el que más referencias tiene respecto a artículos de generación de mutantes o su uso, aunque siguen dentro de los conocidos. Muskit de Arcaini, QMutPy de Rui, MutTG de Arcaini, MorphQ de Paltenghi (testing de Qiskit) y los anteriores.

MutTG: Se focaliza en cómo identificar mutantes equivalentes, no habla de cómo generan los mutantes. Si bien es cierto, me he metido en el código y usan una librería llamada jmetal, ¿Sabeis algo de esta librería?
\section{Methodology}
\label{Ch1.1:TPlat}

\section{Goals}
\label{Ch1.2:TQP}
