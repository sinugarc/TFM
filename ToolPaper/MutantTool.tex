% This is samplepaper.tex, a sample chapter demonstrating the
% LLNCS macro package for Springer Computer Science proceedings;
% Version 2.21 of 2022/01/12
%
\documentclass[runningheads]{llncs}
%
\usepackage[T1]{fontenc}
% T1 fonts will be used to generate the final print and online PDFs,
% so please use T1 fonts in your manuscript whenever possible.
% Other font encondings may result in incorrect characters.
%
\usepackage{graphicx}
% Used for displaying a sample figure. If possible, figure files should
% be included in EPS format.
%
% If you use the hyperref package, please uncomment the following two lines
% to display URLs in blue roman font according to Springer's eBook style:
%\usepackage{color}
%\renewcommand\UrlFont{\color{blue}\rmfamily}
%
\begin{document}
%
\title{Contribution Title\thanks{Supported by organization x.}}
%
%\titlerunning{Abbreviated paper title}
% If the paper title is too long for the running head, you can set
% an abbreviated paper title here
%
\author{Sinhué García Gil \and
Luis Llana\inst{1}\and
José Ignacio Requeno Jarabo\inst{1}}
%
\authorrunning{S. García et al.}
% First names are abbreviated in the running head.
% If there are more than two authors, 'et al.' is used.
%
\institute{Software Systems and Computation\\Complutense
University of Madrid\\C/Prof. José García Santesmases, 9,
28040 Madrid, Spain}
%
\maketitle           % typeset the header of the contribution
%
\begin{abstract}
Quantum computing has been on the rise in recent years, evidenced by a surge in publications on quantum software engineering and testing. Progress in quantum hardware has also been notable, with the introduction of impressive systems like Condor boasting 1121 qubits, and IBM Quantum System Two, which employs three 133-qubit Heron processors. As this technology edges closer to practical application, ensuring the efficacy of our software becomes imperative. Mutation testing, a well-established technique in classical computing, emerges as a valuable approach in this context. In our paper, we aim to introduce a mutation tool tailored for quantum programs in Qiskit, leveraging Qiskit's inherent class structure. Departing from the exhaustive processes of previous works \cite{mendiluze2021muskit} \cite{fortunato2022qmutpy}, we propose a randomised approach and the capability for marking immutable positions within the circuit. This feature facilitates the preservation of program structure, crucial for future applications.

\keywords{Quantum computing \and Mutation tool \and Testing.}
\end{abstract}
%
%
%
\section{Introduction}

Quantum computing and testing
\section{Background and related work}

Background on testing?\newline

Different observations about the available tools and why we decide to create a new one.
\section{Tool development}

\subsection{Insights and expected outcome (RQ)}

Qiskit structure in use, near QASM.

\subsection{Mutant generation}

Gate equivalence\newline

Mutation operators\newline

Randomness (for mutants and inputs?)\newline

Coverage: Study of the probability of no mutating a gate (No using code, so we are mutating over gates)
        May need to mention about the possibility of having to insert in all gaps, therefore we would need to produce more mutants and observe the minimum to ensure the right coverage with a certain significance.

\section{Empirical experiments}

\subsection{Empirical set up and metrics}

\subsection{Results}

Compare vs Muskit and QMutPy\newline

Answer RQ

\subsection{Threats to validity}

\subsubsection{Threats to external validity}
\subsubsection{Threats to internal validity}
\subsubsection{Threats to construct validity}
\input{ToolPaper/conclusion}

\subsubsection{Acknowledgements} Please place your acknowledgements at
the end of the paper, preceded by an unnumbered run-in heading (i.e.
3rd-level heading).

%
% ---- Bibliography ----
%
% BibTeX users should specify bibliography style 'splncs04'.
% References will then be sorted and formatted in the correct style.
%
\bibliographystyle{splncs04}
\bibliography{references}


\end{document}
